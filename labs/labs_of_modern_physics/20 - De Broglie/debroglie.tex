\documentclass[a4paper,12pt,titlepage]{article}

%Εισαγωγή γλωσσικής υποστήριξης
%ελληνικό hyphenation

\usepackage[cm-default]{fontspec}
\usepackage{xunicode}
\usepackage{xltxtra}
\usepackage{xgreek}
\usepackage[colorlinks]{hyperref}
\usepackage{enumerate}
\usepackage{amsmath}
\usepackage{float} %Places the float at precisely the location in the LaTeX code.
%η γραμματοσειρά
%\setmainfont[Mapping=tex-text]{Times New Roman} %απλοποιημένο σε σχέση με το άρθρο
\setmainfont[Mapping=tex-text]{Linux Libertine} 
%page orientation
\usepackage{a4wide}
\voffset = -0.5in
\textheight = 664pt

%Άλλα χρήσιμα πακέτα
\usepackage{graphicx} 	%εισαγωγή εικόνων jpg/png κλπ
\usepackage{listings}
\lstset
{
commentstyle=\textit,
captionpos=b,
breakatwhitespace=true,
showstringspaces=false,
breaklines=true,
keywordstyle=\color{black}\bfseries,
float=htb,
frame=single
}

%απενεργοποίηση του indent στις νέες παραγράφους
\parindent=0in

%macro που δίνει το μέγιστο επιτρεπτό μέγεθος σε μια εικόνα χωρίς να παραβιάζει τα όρια του LaTeX
\makeatletter
\def\maxwidth{
\ifdim\Gin@nat@width>\linewidth
\linewidth
\else
\Gin@nat@width
\fi
    }
\makeatother

\title{1η Άσκηση\\kn}
\author{Axil}
\date{\today}

\begin{document}

\pagestyle{headings}    %αρίθμηση στο πάνω μέρος της σελίδας

%\maketitle
\begin{titlepage}
\begin{center}
\includegraphics[width=50mm]{pyrforos.pdf}\\[0.5cm]
\textbf{\LARGE ΕΘΝΙΚΟ ΜΕΤΣΟΒΙΟ ΠΟΛΥΤΕΧΝΕΙΟ}\\
\textrm{\Large Σχολή Εφαρμοσμένων Μαθηματικών και Φυσικών Επιστημών}\\[2.0cm]
\Huge{Εργαστήριο Σύγχρονης Φυσικής}\\
\Large{\textit{5o εξάμηνο, ΣΕΜΦΕ}}\\[2.0cm]
\Large{\textit{\textbf{Περίθλαση ηλεκτρονίων}}}\\[5.0cm]
\normalsize
\begin{minipage}{0.49\textwidth}
\begin{flushleft}
\textbf{Πιπινέλλης Αχιλλέας}, 09103163
\end{flushleft}
\end{minipage}
\begin{minipage}{0.49\textwidth}
\begin{flushright}
\textbf{\textit{Η/Μ Παράδοσης:} 21 Νοεμβρίου 2011}
\end{flushright}
\end{minipage}
%\maketitle

\vfill
%bottom of the page
{Αθήνα, 2011}

\end{center}
\end{titlepage}


\section{Σκοπός του πειράματος}
Σκοπός του πειράματος είναι η επιβεβαίωση της υπόθεσης του De Broglie για τα υλικά κύματα. Για να το πετύχουμε θα μελετήσουμε το φαινόμενο της περίθλασης των ηλεκτρονίων μέσα από ένα φύλλο γραφίτη.


\section{Λίγη θεωρία}




\section{Μέθοδος πειράματος}




\section{Εκτέλεση πειράματος}

\begin{table}[H]
\begin{center}
\begin{tabular}{ | c | c | c | c|}
\hline
$V (KV)$ & $r_{1} (cm)$ & $r_{2} (cm)$ & $\lambda ($\AA$ ) $ \\ \hline
4.8 & 1.3 & 2.2 & 0.177 \\ \hline
4.5 & 1.4 & 2.3 & 0.183 \\ \hline
4.0 & 1.5 & 2.3 & 0.194 \\ \hline
3.5 & 1.5 & 2.5 & 0.207 \\ \hline
3.0 & 1.6 & 2.8 & 0.224 \\ \hline
2.5 & 1.8 & 3.0 & 0.245 \\ \hline
2.0 & 2.0 & 3.2 & 0.274 \\ \hline
\end{tabular}
\end{center}
\caption{}
\end{table}


\subsection{Σφάλματα}

\begin{tabular}{l | l}
\hline
ακτίνα $r$ & $2mm$ \\
τάση $V$  &  $100V$\\
μήκος κύματος $\lambda$ & \\
\hline
\end{tabular}

\end{document}
