\documentclass[a4paper,11pt,titlepage]{article}

%Εισαγωγή γλωσσικής υποστήριξης
%ελληνικό hyphenation

\usepackage[cm-default]{fontspec}
\usepackage{xunicode}
\usepackage{xltxtra}
\usepackage{xgreek}
\usepackage[colorlinks]{hyperref}
\usepackage{enumerate}
\usepackage{amsmath}
\usepackage{float} %Places the float at precisely the location in the LaTeX code.
%η γραμματοσειρά
%\setmainfont[Mapping=tex-text]{Times New Roman} %απλοποιημένο σε σχέση με το άρθρο
\setmainfont[Mapping=tex-text]{Linux Libertine} 
%page orientation
\usepackage{a4wide}
\voffset = -0.5in
\textheight = 664pt

%Άλλα χρήσιμα πακέτα
\usepackage{graphicx} 	%εισαγωγή εικόνων jpg/png κλπ
\usepackage{listings}
\lstset
{
commentstyle=\textit,
captionpos=b,
breakatwhitespace=true,
showstringspaces=false,
breaklines=true,
keywordstyle=\color{black}\bfseries,
float=htb,
frame=single
}

%απενεργοποίηση του indent στις νέες παραγράφους
\parindent=0in

%macro που δίνει το μέγιστο επιτρεπτό μέγεθος σε μια εικόνα χωρίς να παραβιάζει τα όρια του LaTeX
\makeatletter
\def\maxwidth{
\ifdim\Gin@nat@width>\linewidth
\linewidth
\else
\Gin@nat@width
\fi
    }
\makeatother

\title{1η Άσκηση\\kn}
\author{Axil}
\date{\today}

\begin{document}

\pagestyle{headings}    %αρίθμηση στο πάνω μέρος της σελίδας

%\maketitle
\begin{titlepage}
\begin{center}
\includegraphics[width=50mm]{pyrforos.pdf}\\[0.5cm]
\textbf{\LARGE ΕΘΝΙΚΟ ΜΕΤΣΟΒΙΟ ΠΟΛΥΤΕΧΝΕΙΟ}\\
\textrm{\Large Σχολή Εφαρμοσμένων Μαθηματικών και Φυσικών Επιστημών}\\[2.0cm]
\Huge{Εργαστήριο Σύγχρονης Φυσικής}\\
\Large{\textit{7o εξάμηνο, ΣΕΜΦΕ}}\\[2.0cm]
\Large{\textit{\textbf{Μελέτη των ακουστικών κυμάτων \\ σε ηχητικό σωλήνα}}}\\[5.0cm]
\normalsize
\begin{minipage}{0.49\textwidth}
\begin{flushleft}
\textbf{Πιπινέλλης Αχιλλέας}, 09103163
\end{flushleft}
\end{minipage}
\begin{minipage}{0.49\textwidth}
\begin{flushright}
\textbf{\textit{Η/Μ Παράδοσης:} 21 Νοεμβρίου 2011}
\end{flushright}
\end{minipage}
%\maketitle

\vfill
%bottom of the page
{Αθήνα, 2012}

\end{center}
\end{titlepage}

\section{Σκοπός}
Στην παρούσα άσκηση θα μελετήσουμε τα στάσιμα ακουστικά κύματα σε έναν ακουστικο σωλήνα. Συγκεκριμένα θα καταγράψουμε την ακουστική πίεση κατά μήκος του σωλήνα όταν το ελεύθερο άκρο του είναι κλειστό και όταν είναι ανοιχτό.

\section{Πείραμα}

\subsection{Πειραματική διάταξη}

Θα χρησιμοποιήσουμε έναν ηχητικό σωλήνα, έναν παλμογράφο, μία γεννήτρια για την τροφοδοσία του μεγαφώνου και ένα μικρόφωνο με τον ενισχυτή του. Το ίδιο κύκλωμα θα χρησιμοποιηθεί για πειράματα με αρμονικά σήματα και πειράματα με κρουστικούς ηχητικούς παλμούς.

\begin{table} [H]
\centering
\begin{tabular}{|c|c|c|c|c}
\hline \rule[-2ex]{0pt}{5.5ex} $x (cm)$ & $f_{1} (Hz)$ κλειστό άκρο & $f_{1} (Hz)$ ανοιχτό άκρο & $f_{2} (Hz)$ κλειστό άκρο & $f_{2} (Hz)$ ανοιχτό άκρο \\ 
\hline \rule[-2ex]{0pt}{5.5ex} 0 & 6.4 & 261 & 134 \\
\hline \rule[-2ex]{0pt}{5.5ex} 5 & 6.0 & 261 & 134 \\
\hline \rule[-2ex]{0pt}{5.5ex} 10 & 5.6 & 265 & 129 \\
\hline \rule[-2ex]{0pt}{5.5ex} 15 & 5.1 & 261 & 134 \\
\hline \rule[-2ex]{0pt}{5.5ex} 20 & 4.6 & 312 & 81 \\
\hline \rule[-2ex]{0pt}{5.5ex} 25 & 396 & 261 & 134 \\
\hline \rule[-2ex]{0pt}{5.5ex} 30 & 390 & 230 & 60 \\
\hline \rule[-2ex]{0pt}{5.5ex} 35 & 396 & 261 & 134 \\
\hline \rule[-2ex]{0pt}{5.5ex} 40 & 388 & 346 & 42 \\
\hline \rule[-2ex]{0pt}{5.5ex} 45 & 396 & 261 & 134 \\
\hline \rule[-2ex]{0pt}{5.5ex} 50 & 385 & 352 & 33 \\
\hline \rule[-2ex]{0pt}{5.5ex} 55 & 396 & 261 & 134 \\
\hline \rule[-2ex]{0pt}{5.5ex} 60 & 389 & 346 & 43 \\
\hline \rule[-2ex]{0pt}{5.5ex} 65 & 396 & 261 & 134 \\
\hline \rule[-2ex]{0pt}{5.5ex} 70 & 392 & 329 & 63\\
\hline \rule[-2ex]{0pt}{5.5ex} 75 & 396 & 261 & 134 \\
\hline \rule[-2ex]{0pt}{5.5ex} 80 & 396 & 305 & 91 \\
\hline \rule[-2ex]{0pt}{5.5ex} 85 & 396 & 261 & 134 \\
\hline \rule[-2ex]{0pt}{5.5ex} 90 & 396 & 261 & 134 \\
\hline 
\end{tabular} 
\caption{}
\end{table}


\end{document}
