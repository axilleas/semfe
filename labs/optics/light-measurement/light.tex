\documentclass[a4paper,11pt,titlepage]{article}

%Εισαγωγή γλωσσικής υποστήριξης
%ελληνικό hyphenation

\usepackage[cm-default]{fontspec}
\usepackage{xunicode}
\usepackage{xltxtra}
\usepackage{xgreek}
\usepackage[colorlinks]{hyperref}
\usepackage{enumerate}
\usepackage{amsmath}
%η γραμματοσειρά
%\setmainfont[Mapping=tex-text]{Times New Roman} %απλοποιημένο σε σχέση με το άρθρο
\setmainfont[Mapping=tex-text]{Linux Libertine} 
%page orientation
\usepackage{a4wide}
\voffset = -0.5in
\textheight = 664pt

%Άλλα χρήσιμα πακέτα
\usepackage{graphicx} 	%εισαγωγή εικόνων jpg/png κλπ
\usepackage{listings}
\lstset{commentstyle=\textit,
captionpos=b,
breakatwhitespace=true,
showstringspaces=false,
breaklines=true,
keywordstyle=\color{black}\bfseries,
float=htb,
frame=single}

%απενεργοποίηση του indent στις νέες παραγράφους
\parindent=0in

%macro που δίνει το μέγιστο επιτρεπτό μέγεθος σε μια εικόνα χωρίς να παραβιάζει τα όρια του LaTeX
\makeatletter
\def\maxwidth{
\ifdim\Gin@nat@width>\linewidth
\linewidth
\else
\Gin@nat@width
\fi
}

\makeatother

\title{1η Άσκηση\\kn}
\author{AxiP}
\date{\today}

\begin{document}

\pagestyle{headings}    %αρίθμηση στο πάνω μέρος της σελίδας

%\maketitle
\begin{titlepage}
\begin{center}
\includegraphics[width=50mm]{pyrforos.pdf}\\[0.5cm]
\textbf{\LARGE ΕΘΝΙΚΟ ΜΕΤΣΟΒΙΟ ΠΟΛΥΤΕΧΝΕΙΟ}\\
\textrm{\Large Σχολή Εφαρμοσμένων Μαθηματικών και Φυσικών Επιστημών}\\[2.0cm]
\Huge{Εργαστήριο Οπτικής\\}
\Large{\textit{5\textsuperscript{o} εξάμηνο, ΣΕΜΦΕ}}\\[2.0cm]
\Large{\textit{\textbf{Μέτρηση της ταχύτητας του φωτός\\Μέτρηση αποστάσεων}}}\\[5.0cm]
\normalsize

\begin{minipage}{0.49\textwidth}
\begin{flushleft}
\textbf{Αχιλλέας Πιπινέλλης}, 09103163
\end{flushleft}
\end{minipage}
\begin{minipage}{0.49\textwidth}
\begin{flushright}
\textbf{\\}
\textit{Η/Μ Παράδοσης:} 22 ιανουαρίου 2012
\end{flushright}
\end{minipage}

%\maketitle

\vfill
%bottom of the page
{Αθήνα, 2012}

\end{center}
\end{titlepage}

\section{Θεωρία}
Η ένταση ενός laser μπορεί να διαμορφωθεί κατάλληλα ώστε να μεταφερθεί οπτική ή
ακουστική πληροφορία. Αυξομειώνοντας την υψηλή τάση στη λυχνία του laser, μία
πληροφορία μπορεί να διαμορφώσει κατα πλάτος τη δέσμη του laser και να
μεταφερθεί σε μεγάλες αποστάσεις. Αυτό το φαινόμενο θα αξιοποιήσουμε εδώ γα να
μετρήσουμε την ταχύτητα του φωτός. Θα χρησιμοποιήσουμε ένα κάτοπτρο σε
γνωστή απόσταση από το laser και μέσω ενώ παλμογράφου θα μετρήσουμε την καθυστέρηση του φωτός για να πάει και
να επιστρέψει από το κάτοπτρο. 

\section{Πείραμα}

Τα εξαρτήματα και τα όργανα που χρσιμοποιήσαμε είναι τα παρακάτω:

Laser He-Ne
Παλμογράφος
Δύο φακοί
Ένα κάτοπτρο

Η απόσταση που θα στηθεί το κάτοπτρο πρέπει να είναι τουλάχιστον 20m. Μπροστά από το laser τοποθετούμε
ένα ημιδιαφανές κάτοπτρο, το οποίο διαχωρίζει τη δέσμη σε 2 νέους συρμούς.  



\subsection{Επεξεργασία μετρήσεων}


\end{document}
