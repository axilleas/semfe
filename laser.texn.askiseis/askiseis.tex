\documentclass[a4paper,11pt,titlepage]{article}

%Εισαγωγή γλωσσικής υποστήριξης
%ελληνικό hyphenation

\usepackage{fontspec}
\usepackage{xunicode}
\usepackage{xltxtra}
\usepackage{xgreek}
\usepackage[colorlinks]{hyperref}
\usepackage{enumerate}
\usepackage{amsmath}
%η γραμματοσειρά
%\setmainfont[Mapping=tex-text]{Times New Roman} %απλοποιημένο σε σχέση με το άρθρο
\setmainfont[Mapping=tex-text]{Linux Libertine} 
%page orientation
\usepackage{a4wide}
\voffset = -0.5in
\textheight = 664pt

%Custom commands
\newcommand{\supscript}[1]{\ensuremath{^{\textrm{#1}}}}
\newcommand{\subscript}[1]{\ensuremath{_{\textrm{#1}}}}
\newcommand{\degrees}{^{\circ}}


%Άλλα χρήσιμα πακέτα
\usepackage{graphicx} 	%εισαγωγή εικόνων jpg/png κλπ
\usepackage{listings}
\lstset{commentstyle=\textit,
captionpos=b,
breakatwhitespace=true,
showstringspaces=false,
breaklines=true,
keywordstyle=\color{black}\bfseries,
float=htb,
frame=single}

%απενεργοποίηση του indent στις νέες παραγράφους
\parindent=0in

%macro που δίνει το μέγιστο επιτρεπτό μέγεθος σε μια εικόνα χωρίς να παραβιάζει τα όρια του LaTeX
\makeatletter
\def\maxwidth{
\ifdim\Gin@nat@width>\linewidth
\linewidth
\else
\Gin@nat@width
\fi
}
\makeatother


\begin{document}

%\pagestyle{headings}    %αρίθμηση στο πάνω μέρος της σελίδας

%\maketitle
\begin{titlepage}
\begin{center}
\includegraphics[width=50mm]{pyrforos.pdf}\\[0.5cm]
\textbf{\LARGE ΕΘΝΙΚΟ ΜΕΤΣΟΒΙΟ ΠΟΛΥΤΕΧΝΕΙΟ}\\
\textrm{\Large Σχολή Εφαρμοσμένων Μαθηματικών και Φυσικών Επιστημών}\\[2.0cm]
\Huge{Φυσική και Τεχνολογία των laser\\}
\Large{\textit{$6^o$} εξάμηνο, ΣΕΜΦΕ}\\[2.0cm]
\Large{\textit{\textbf{Ασκήσεις}}}\\[5.0cm]
\normalsize

\vfill
%bottom of the page
{Αθήνα, 2012}

\end{center}
\end{titlepage}

\section{Άσκηση 1}
Υπολογίστε τη συχνότητα($Hz$), τον κυματαριθμό($cm^{-1}$) και την ενέργεια ενός φωτονίου μήκους κύματος $\lambda=1{\mu}m$ στο κενό.

\subsection*{\underline{Λύση:}}

Στο κενό έχουμε 

\begin{equation}
 \lambda\nu=c
\end{equation}
 και επομένως

\begin{equation}
 \nu=\frac{\lambda}{c}=3\times10^{14} Hz
\end{equation}
άρα
\begin{equation}
 E=h\nu=1.99\times10^{-19}J
\end{equation}

Λόγω του ότι $1eV=1.6\times10^{-19}J$ προκύπτει ότι $E=1.24eV$, δηλαδή ίση με την κινητική ενέργεια ενός ηλεκτρονίου που έχει επιταχυνθεί  από μία διαφορά δυναμικού $1.24eV$. Το αντίστροφο μήκος κύματος (κυματαριθμός) είναι $\widetilde{u}=\nu/c$ και χρησιμοποιείται συχνά με την έννοια της συχνότητας και εκφυλίζεται σε $cm^{-1}$, δηλαδή  

\begin{equation}
\widetilde{N}=\frac{\nu}{c}=10^4cm^{-1} 
\end{equation}
\begin{quote}
\textbf{Σημείωση:} Ο τύπος μετατροπής ενέργειας $eV$ και μήκους κύματος $\lambda$ (${\mu}m$) είναι

\begin{equation}
 E(eV)=\frac{1.24}{\lambda({\mu}m)}
\end{equation}

Ο κυματαριθμός είναι λοιπόν
\begin{equation}
 \widetilde{N}=\nu/c=\frac{E}{hc}
\end{equation}

$1cm^{-1}\sim1.24\times10^{-4}eV$\\ $1eV\sim8065.5 cm^{-1}$
\end{quote}
\newpage
\section{Άσκηση 2}

Υπολογίστε σε κυματαριθμούς την ενέργεια ${\Delta}E=kT$, όπου $k$ η σταθερά Boltzmann και $T$ η απόλυτη θερμοκρασία ($300K$).

\subsection*{\underline{Λύση:}}

Η θερμική ενέργεια των 300K δίνεται από τον τύπο
\begin{equation}
 kT=4.14\times10^{-21} J
\end{equation}

Η σχέση κυματαριθμού-ενέργειας είναι
\begin{equation}
 \widetilde{u}=\frac{\nu}{c}=\frac{E}{hc}=\frac{kT}{hc}=208.5 cm^{-1}
\end{equation}

\section{Άσκηση 3 (1.3 από λυσάρι)}

Αν τα επίπεδα 1 και 2 ενός κβαντικού συστήματος είναι διαχωρισμένα από ενέργεια $E_2-E_1$ τέτοια που η αντίστοιχη συχνότητα μετάπτωσης εμπίπτει στο μέσο της ορατηής περιοχής, υπολογίστε το λόγο των πληθυσμών των δύο επιπέδων σε θερμική ισορροπία σε θερμοκρασία δωματίου.

\subsection*{\underline{Λύση:}}
Σε περίπτωση θερμικής ισορροπίας, οι πληθυσμοί επιπέδων περιγράφονται από τη στατιστική Boltzmann. Παίρνοντας $\lambda=0.55{\mu}m$, σαν το μέσο της ορατής περιοχής, αντιστοιχεί σε συχνότητα $18.181cm^{-1}$. Η εξίσωση της στατιστικής Boltzmann είναι:

\begin{equation}
\frac{N^e_2}{N^e_1}=exp[-\frac{E_2-E_1}{kT}]=1.1\times10^{-38}
\end{equation}

όπου $kT=208cm^{-1}$, $E_2-E_1=18.181cm^{-1}$

\begin{quote}
 \textbf{Σημείωση:} 
\begin{itemize}
 \item Όταν βλέπουμε άσκηση με θερμική ισορροπία, σκεφτόμαστε αμέσως τη στατιστική Boltzmann
\begin{equation}
exp[-\dfrac{E_2-E_1}{kT}]=\dfrac{N_2}{N_1}
\end{equation}
 \item Στην περιοχή του ορατού παίρνουμε όποιο $\lambda$ μας βολεύει ($0.5-1{\mu}m$).
\end{itemize}
\end{quote}

\newpage
\section{Άσκηση 4 (1.4 από λυσάρι)}

Σε κατάσταση θερμικής ισορροπίας ($T=300K$), ο λόγος των πληθυσμών των επιπέδων $N_2/N_1$ για κάποιο ιδιαίτερο ζεύγος επιπέδων δίνεται από το $1/e$. Υπολογίστε τη συχνότητα για αυτή τη μετάπτωση. Σε ποια περιοχή Η.Μ. φάσματος εμπίπτει αυτή η συχνότητα;

\subsection*{\underline{Λύση:}}

Από την εξίσωση της στατιστικής Boltzmann στην περίπτωση που $N_2/N_1=1/e$ έχουμε,

\begin{equation}
 E_2-E_1=kT
\end{equation}
η οποία για θερμοκρασία δωματίου δίνει:

\begin{equation}
 E_1-E_2=208cm^{-1}
\end{equation}

Αυτός ο ενεργειακός διαχωρισμός αντιστοιχεί σε $\lambda=49{\mu}m$ (υπέρυθρο, όχι μεσαίο υπέρυθρο).
\newpage
\section{Άσκηση 5}

Προσδιορίστε το λόγο των πληθυσμών, σε θερμική ισορροπία δύο επιπέδων που απέχουν κατά ενέργεια ${\Delta}E$ ίση με:
\begin{enumerate}
 \item$10^{-4}eV$ τιμή που ισοδυναμεί με την απόσταση δύο περιστροφικών επιπέδων πολλών μορίων 
 \item$5\times10^{-2}eV$ τιμή που ισοδυναμεί με τα μοριακά δονητικά επίπεδα
 \item$3eV$ τιμή που αντιστοιχεί στην τάξη μεγέθους της ηλεκτρονικής διέγερσης ατόμων
\end{enumerate}
Θεωρήστε ότι τα δύο επίπεδα έχουν τον ίδιο εκφυλισμό και ότι η θερμοκρασία είναι 100Κ, 300Κ (δωματίου) και 1000Κ.
\subsection*{\underline{Λύση:}}
Ο λόγος των πληθυσμών δύο επιπέδων σε θερμική ισορροπία που απέχουν κατά \\${\Delta}E=E_2-E_1>0$ δίνεται από την:

\begin{equation}
\dfrac{N_2}{N_1}=\dfrac{g_2}{g_1}exp[-\dfrac{{\Delta}E}{kT}]
\end{equation}
\\\\

\begin{tabular}{l | l}
\hline
 $N_1,N_2$ & οι πληθυσμοί\\
 $g_1,g_2$ & οι εκφυλισμοί των δύο επιπέδων\\
\hline
\end{tabular}
\\\\
Από την παραπάνω εξίσωση και ξέροντας πως $g_1=g_2$ έχουμε:

\begin{table}[h!]
\begin{center}
    \begin{tabular}{ | c | c | c | c | p{5cm} |}
    \hline
     ${\Delta}E (eV)$	& $T=100K$		& $T=300K$ 		& $T=1000K$		& Σχόλια\\ \hline
     $10^{-4}$		& $0.9885$		& $0.9962$		& $0.9988$		& Ίδιοι πληθυσμοί, εύκολη αντιστροφή\\ \hline
     $5\times10^{-2}$	& $9\times10^{-3}$	& $1.45\times10^{-1}$	& $5.6\times10^{-1}$	& Μερικοί πληθυσμοί στο $2^ο$ επίπεδο, αρκετή προσπάθεια για αντιστροφή. Πρέπει να ξεφύγει από θερμική ισορροπία και να πάει σε υψηλές θερμοκρασίες\\ \hline
     $3$			& $5\times10^{-184}$	& $8\times10^{-49}$	& $8\times10^{-16}$	& Πολύ δύσκολη η αντιστροφή πληθυσμών. Αμελητέοι πληθυσμοί στο $2^ο$ επίπεδο.\\ \hline
    \end{tabular}
\end{center}
\end{table}
\newpage
\section{Άσκηση 6 (1.6 από λυσάρι)}

Η δέσμη ενός laser ρουβιδίου ($\lambda=0.694{\mu}m$), στέλνεται προς τη σελήνη αφού περάσει από τηλεσκόπιο διαμέτρου $1m$. Υπολογίστε τη διάμετρο $D_m$ της δέσμης στη σελήνη υποθέτοντας πως η δέσμη έχει τέλεια χωρική συμφωνία (η απόσταση μεταξύ γης-σελήνης είναι περίπου $384.000km$).
\subsection*{\underline{Λύση:}}

Το μέγεθος της δέσμης στη σελήνη ορίζεται από την απόσταση δέσμης $\theta_d$. Σύμφωνα με την παρακάτω σχέση και θέτοντας $\beta=1$, βρίσκουμε:
\begin{equation}
 \theta_d=\frac{\beta\lambda}{D}=0.694\times10^{-6} rad
\end{equation}

όπου D είναι η διάμετρος του τηλεσκοπίου. Έτσι η διάμετρος της δέσμης στη σελήνη δίνεται από:
\begin{equation}
 D_m=2Ltan{\theta_d}=2L\theta_d=532m
\end{equation}
όπου L η απόσταση γης-σελήνης.

\textbf{Σημείωση:} Η διάμετρος του τελευταίου οπτικού στοιχείου που περνάει η δέσμη laser λέγεται διάμετρος πηγής.

\section{Άσκηση 7}

Δέσμη laser ισχύος $P_l=10W$ εστιάζει σε φωτεινή κηλίδα διαμέτρου $d=1mm$, πάνω σε μία απόλυτα απορροφητική επιφάνεια-στόχο. Υπολογίστε την πίεση ακτινοβολίας $P_t$ πάνω στο στόχο.

\subsection*{\underline{Λύση:}}

Η ένταση της δέσμης είναι ίση με 
\begin{equation}
I=4P_l/{\pi}d_2=1.27\times10^{13} Wm^{-2}
\end{equation}
όπου ${\pi}d^2/4$ το εμβαδό της εστιασμένης δέσμης. Επομένως η πίεση της ακτινοβολίας πάνω στο στόχο είναι:
\begin{equation}
 P_t=\frac{I}{c}=4.2\times10^4Pa=0.42 bar
\end{equation}
η οποία είναι συγκρίσιμη με την ατμοσφαιρική.
\newpage
\section{Άσκηση 8 (Χρόνος και μήκος συμφωνίας ασύμφωνης μονοχρωματικής ακτινοβολίας)}
Χρησιμοποιούμε ένα φίλτρο συμβολής με ζώνη διέλευσης $10nm$ στα $500nm$ για να πετύχουμε μονοχρωματικό φως από πηγή λευκού φωτός. Υπολογίστε το χρόνο και το μήκος συμφωνίας του μονοχρωματικού φωτός.

\subsection*{\underline{Λύση:}}

Αν ο χρόνος συμφωνίας ΗΜ κύματος είναι $\tau_0$ τότε το εύρος ζώνης αντίστοιχα είναι $\Delta\nu_0\simeq 1/\tau_0$. Επομένως αντίστροφα αν το εύρος ζώνης είναι $\Delta\nu_0$ ο αντίστοιχος χρόνος συμφωνίας είναι $\tau_0=1/\Delta\nu_0$. Χρησιμοποιώντας τη σχέση $c=\lambda\nu$ έχουμε:
\begin{equation}
 |\Delta\nu|=\frac{c}{\lambda^2}|\Delta\lambda|
\end{equation}

\begin{equation}
 \tau_0=\frac{1}{\Delta\nu}=\frac{\lambda^2}{c\Delta\lambda}=83fs
\end{equation}

Το μήκος συμφωνίας είναι:
\begin{equation}
 \lambda_c=c\tau_0=\frac{\lambda^2}{\Delta\lambda}=2.5\times10^{-5}m
\end{equation}


\section{Άσκηση 9}

Laser με οπτικό αντηχείο αποτελείται από δύο κάτοπτρα με ακτίνα καμπυλότητας $R=R_1=R_2=200mm$. Λόγω κατασκευαστικών σφαλμάτων οι πραγματικές ακτίνες καμπυλότητας είναι $R_1=R+{\Delta}R$  και $R_2=R-{\Delta}R$, όπου ${\Delta}R=3mm$. Τα laser λειτουργεί σωστά όταν τα κάτοπτρα τοποθετηθούν σε μικρότερη ή μεγαλύτερη απόσταση από την ομοεστιακή θέση. Εξηγήστε αυτή την πειραματική διαπίστωση και προσδιορίστε ακριβώς την απόσταση για την οποία το laser αρχίζει να λειτουργεί.

\subsection*{\underline{Λύση:}}
Για $L=R=200mm$ οι παράμετροι σταθερότητας του αντηχείου είναι:


\begin{equation}
 g_1=1-\frac{R}{R+{\Delta}R}>0
\end{equation} 

 \begin{equation}
 g_2=1-\frac{R}{R-{\Delta}R}<0
\end{equation} 

άρα για $g_1g_2<0$, το οπτικό αντηχείο είναι ασταθές. Θα πρέπει λοιπόν:

\begin{equation}
 g_1g_2>0\Rightarrow(1-\frac{R}{R+{\Delta}R})(1-\frac{R}{R-{\Delta}R})>0\Rightarrow L^2-2RL+({R+{\Delta}R})({R-{\Delta}R})>0
\end{equation}

Η παραπάνω ανίσωση ικανοποιείται για \\

$L>R+\Delta R$ \\
$L<R-\Delta R$ \\

Επομένως θα πρέπει τα κάτοπτρα να μετακινηθούν κατά $3mm$ προς τα μέσα ή προς τα έξω από την ομοεστιακή σχέση.

\section{Άσκηση 10}

Δείξτε ότι η πίεση ακτινοβολίας μίας φωτεινής δέσμης έντασης $I$ που προσπίπτει κάθετα σε μία απόλυτα απορροφητική επιφάνεια είναι $I/c$. Η ορμή φωτονίου συχνότητας $V$ δίνεται $q=(h/2\pi)/k$ και $k=2\pi\nu/c$.

\subsection*{\underline{Λύση:}}

Η συνολική ορμή διατηρείται και κάθε φωτόνιο μεταφέρει την ορμή του στην επιφάνεια. Αν $\Phi$ η ροή των φωτονίων, τότε η συνολική ορμή σε επιφάνεια ${\Delta}s$ σε χρόνο $\Delta t$ είναι:

\begin{equation}
 Q=\Phi q\Delta s\Delta t
\end{equation}

Η δύναμη που ασκείται στη $\Delta s$ είναι $Q=F\Delta t$ και η πίεση:

\begin{equation}
 P=\dfrac{F}{\Delta s}=\dfrac{Q}{\Delta s\Delta t}=\Phi q=\dfrac{\Phi h\nu}{c}
\end{equation}

Άρα, αφού $I=\Phi h\nu$, έχουμε ότι $P=I/c$.

\textbf{Σημείωση:}
\begin{itemize}
 \item Αν η επιφάνεια είναι ανακλαστική τότε $P=2I/c$ και το φωτόνιο κατά την ανάκλαση έχει φορτίο $\Delta q=q-(-q)=2q$.
 \item Αν έχουμε πρόσπτωση υπό γωνία $\theta$ τότε η ορμή είναι ίση με 
\begin{equation}
Q=\Phi q\Delta \cos\theta \Delta t 
\end{equation}
και άρα η πίεση ισούται με
\begin{equation}
 P=\dfrac{\Phi \cos\theta}{\Delta t}=\Phi q\cos^2\theta=\dfrac{I}{c\cos^2\theta}
\end{equation}
\end{itemize}


\section{Άσκηση 11} 

Έχουμε ενεργό υλικό μήκους $5cm$  με συντελεστή ενίσχυσης $g_0=5m^{-1}$ και ένταση κορεσμού $5Wm^{-2}$. Μονοχρωματική ΗΜ ακτινοβολία περνάει μέσα  από το ενεργό υλικό με ένταση $10Wm^{-2}$. Υπολογίστε την ένταση εξόδου.

\subsection*{\underline{Λύση:}}

Η αύξηση της ροής $F$ των φωτονίων μετά από καθυστέρηση $dz$ είναι $dF=gFdz$. Επειδή $F=I/h\nu$ ισχύει ότι:
\begin{equation}
 dI=gIdz
\end{equation}

όπου $g$ ο συντελεστής ενίσχυσης και υπολογίζεται από την:
\begin{equation}
 g=\frac{g_0}{1+\dfrac{I}{I_s}}
\end{equation}

όπου $I_s$ η ένταση κορεσμού.
\\\\
Από τις παραπάνω 2 εξισώσεις έχουμε:

\begin{equation}
(\dfrac{1}{I}+\dfrac{1}{I_s})dI=g_0dz
\end{equation}

Για $I=I_0$ $(z=0)$, έχουμε:
\begin{equation}
 I=I_0\exp ( g_0\ell-\dfrac{I-I_0}{I_s} )
\end{equation}

όπου $\ell$ το μήκος του ενεργού υλικού. \\
Αν ο κορεσμός είναι αμελητέος, στην έξοδο θα έχουμε:

\begin{equation}
 I=I_0\exp(g_0\ell)=12.84Wm^{-2}
\end{equation}


\section{Άσκηση 12}

Να διερευνηθούν οι δυνατότητες μέγιστης αντίστασης και ακρίβειας μέτρησης ενός μετρητή αποστάσεων laser διαφοράς φάσης, με κρύσταλλο χαλαζία μέγιστης συχνότητας $f_{max}=4.433MHz$ και ελάχιστης συχνότητας $f_{min}=80Hz$.

\subsection*{\underline{Λύση:}}

Σε μία τέτοια διάταξη, ο χρόνος που μεσολαβεί από την εκπομπή μέχρι τη λήψη της ακτινοβολίας laser ισούται με $t=2r/c$, όπου r η μετρούμενη απόσταση. Η αντίστοιχη μετατόπιση φάσης είναι $\phi=2\pi ft$, όπου $f$ η συχνότητα του κρυστάλλου. Επομένως

\begin{equation}
 \phi= \frac{4\pi rf}{c}
\end{equation}

\begin{itemize}
 \item Για $4.433MHz$ και πχ $r=10m$
\end{itemize}

\begin{equation}
 \phi=\frac{4\pi \times10m \times 4.433\times10^{-6}s^{-1}}{3\times10^8ms^{-1}}=0.5911\times\pi=106\degrees
\end{equation}

Λόγω του τύπου του κυκλώματος που έχουμε σε αυτές τις διατάξεις, η μέγιστη τιμή της $\phi$ είναι $\phi=\pi=180\degrees$. Επομένως η μέγιστη απόσταση που μπορεί να μετρήσει κανείς στα $4.433MHz$ είναι:

\begin{equation}
 r_{max}=\frac{\phi c}{4\pi f}=\frac{\pi\times 3\times10^8ms^{-1}}{4\pi\times 4.433\times10^{6} s^{-1}}=16.9m
\end{equation}

\begin{itemize}
 \item Για $80Hz$ και πχ $r=10m$
\end{itemize}

$\phi=\pi=180\degrees$ και 

\begin{equation}
 r^{'}_{max}=\frac{\phi c}{4\pi f}=\frac{\pi\times 3\times10^8ms^{-1}}{4\pi\times 80s^{-1}}=937.5 km
\end{equation}

\begin{itemize}
 \item Ακρίβεια
\end{itemize}

Αν η ελάχιστη ανιχνεύσιμη διαφορά φάσης είναι $\Delta\phi=\pm1/4\degrees$ και η μέγιστη συχνότητα $f_{max}=4.433MHz$ τότε η ελάχιστη ανιχνεύσιμη απόσταση είναι η 

\begin{equation}
 \Delta r=\frac{\Delta tc}{2}=\frac{\Delta\phi c}{4\pi f}=\frac{\pm 0.25\degrees\times 3\times10^8ms^{-1}}{4\times 180\degrees \times 4.433\times10^6s^{-1}}=\pm\frac{75}{3192}=2.3cm
\end{equation}

Αν η ελάχιστη ανιχνεύσιμη διαφορά φάσης είναι $\Delta\phi=\pm1/4\degrees$ και η ελάχιστη συχνότητα $f_{min}=80Hz$ τότε η ελάχιστη ανιχνεύσιμη απόσταση είναι η 

\begin{equation}
 \Delta r=\frac{\Delta tc}{2}=\frac{\Delta\phi c}{4\pi f}=\frac{\pm 0.25\degrees\times 3\times10^8ms^{-1}}{4\times 180\degrees \times 80s^{-1}}=\pm\frac{0.75\times10^8m}{57600}=1.3km
\end{equation}

\section{Άσκηση 13}

Ο πομπός laser τηλεμέτρου ηχούς παλμών εκπέμπει ενέργεια $W_T$. Να βρεθεί η επιστρέφουσα στον ανιχνευτή του τηλεμέτρου ενέργεια $W_r$ μετά από ανάκλαση σε στόχο σε απόσταση $R$.

\subsection*{\underline{Λύση:}}



%\section{Άσκηση 14}

%\subsection*{\underline{Λύση:}}

%\section{Άσκηση 15}

%\subsection*{\underline{Λύση:}}

%\section{Άσκηση 16}

%\subsection*{\underline{Λύση:}}

%\section{Άσκηση 17}

%\subsection*{\underline{Λύση:}}

%\section{Άσκηση 18}

%\subsection*{\underline{Λύση:}}

%\section{Άσκηση 19}

%\subsection*{\underline{Λύση:}}

%\section{Άσκηση 20}

%\subsection*{\underline{Λύση:}}

\newpage
\section{Θέμα 1 (Ρυμοί αυθόρμητης και εξαναγκασμένης εκπομπής)}
Για ένα σύστημα σε θερμική ισορροπία υπολογίστε τη θερμοκρασία στην οποία εξισώνονται οι ρυθμοί αυθόρμητης και εξαναγκασμένης εκπομπής για το μήκος κύματος $500nm$. Επίσης να βρεθεί το μήκος κύματος στο οποίο οι ρυθμοί εξισώνονται σε $T=4000K$.

\subsection*{\underline{Λύση:}}

Ο λόγος του ρυθμού αυθόρμητης εκπομπής Α προς το ρυθμό εξαναγκασμένης εκπομπής W, δίνεται από:
\begin{equation}
 R=\dfrac{A}{W}=\dfrac{A}{B\rho{_\nu{_0}}}
\end{equation}
όπου $\rho{_\nu{_0}}$ η πυκνότητα της ενέργειας. Ο A/B δίνεται από την εξίσωση του Einstein:
\begin{equation}
 \frac{A}{B}=\frac{8{\pi}hN_0^3n^3}{c^3}
\end{equation}

H πυκνότητα της ενέργειας δίνεται από τη εξίσωση του Planck:
\begin{equation}
 \rho_\nu=\frac{8\pi\nu^2}{c^3}\frac{h\nu}{exp(h\nu/kT)-1}
\end{equation}
Έτσι έχουμε:
\begin{equation}
 R=exp[\frac{N_0}{kT}-1]=exp[\frac{hc}{kT\lambda}]-1
\end{equation}

Επομένως η θερμοκρασία για την οποία έχουμε $R=1$ είναι και μήκος κύματος $\lambda=500nm$:
\begin{equation}
 T=\dfrac{hc}{k{\lambda}ln2}=41562K
\end{equation}

Το μήκος κύματος στο οποίο οι ρυθμοί εξισώνονται σε $T=4000K$, είναι:
\begin{equation}
 \lambda=\frac{hc}{kTln2}=5.2{\mu}m
\end{equation}











\end{document}
